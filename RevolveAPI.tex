\documentclass[sigconf]{acmart}

\usepackage{booktabs} % For formal tables


% Copyright
%\setcopyright{none}
%\setcopyright{acmcopyright}
%\setcopyright{acmlicensed}
\setcopyright{rightsretained}
%\setcopyright{usgov}
%\setcopyright{usgovmixed}
%\setcopyright{cagov}
%\setcopyright{cagovmixed}


% DOI
\acmDOI{10.475/123_4}

% ISBN
\acmISBN{123-4567-24-567/08/06}

%Conference
\acmConference[SC17]{SuperComputing}{November 2017}{Denver, Colorado, USA} 
\acmYear{1997}
\copyrightyear{2017}

\acmPrice{15.00}


\begin{document}
\title{High-level python abstractions for data management in inversion problems}
%\titlenote{Produces the permission block, and
%  copyright information}
\subtitle{Checkpointing, done beautifully.}
%\subtitlenote{The full version of the author's guide is available as
%  \texttt{acmart.pdf} document}


\author{Navjot Kukreja}
\affiliation{%
  \institution{Imperial College London}
  \streetaddress{Kensington}
  \city{London}
  \postcode{SW7 2AZ}
  \country{UK}}
\author{Jan H\"uckelheim}
\affiliation{%
  \institution{Imperial College London}
  \streetaddress{Kensington}
  \city{London}
  \postcode{SW7 2AZ}
  \country{UK}}
\author{Michael Lange}
\affiliation{%
  \institution{Imperial College London}
  \streetaddress{Kensington}
  \city{London}
  \postcode{SW7 2AZ}
  \country{UK}}
\author{Who Else}
\affiliation{%
  \institution{Imperial College London}
  \streetaddress{Kensington}
  \city{London}
  \postcode{SW7 2AZ}
  \country{UK}}

\renewcommand\shortauthors{Kukreja, N. et al}

\begin{abstract}
The computation of adjoints in optimisation and inverse design problems requires storing intermediate data. The available memory size sets a limit to this, and necessitates recomputation in some instances. Revolve checkpointing offers an optimal schedule that trades computational cost for smaller memory footprints. Integrating Revolve into a modern python HPC code and combining it with code generation is not straightforward. We present an API that makes checkpointing accessible from a DSL-based code generation environment and present a benchmark study. TODO.
\end{abstract}

%
% The code below should be generated by the tool at
% http://dl.acm.org/ccs.cfm
% Please copy and paste the code instead of the example below. 
%
\begin{CCSXML}
<ccs2012>
<concept>
<concept_id>10011007.10011074.10011075.10011077</concept_id>
<concept_desc>Software and its engineering~Software design engineering</concept_desc>
<concept_significance>500</concept_significance>
</concept>
</ccs2012>
\end{CCSXML}

\ccsdesc[500]{Software and its engineering~Software design engineering}
%
% End generated code
%


\keywords{HPC, Code generation, API, Checkpointing, Adjoint, Inverse Problems}


\maketitle

\section{Introduction}
Motivation: Why do we need to connect revolve and a python seismic inversion code? What is there to learn for others? How does our work help readers and the world?

\section{Seismic inversion}
Some context on seismic imaging, the PDEs, the method. Why use adjoints. Why is this problem time-dependent. Why do we need a lot of data.

\section{Devito}
Code generation, references to previous papers about Devito. Why not be brave and write this in Fortran.

\section{Revolve}
How does revolve work and save memory. How do others use it, why is this painful.

\section{API: Connecting Revolve with Devito}
This is the core of the paper.

\section{Test case, Results}
A nice test case that can be scaled up in size easily. Timings for different mesh sizes, with different amounts of memory set as the upper limit.

\section{Conclusions}
This work was highly successful, but more work could be done.

\begin{acks}
  The authors are very grateful.
\end{acks}


\bibliographystyle{ACM-Reference-Format}
\bibliography{sample-bibliography} 

\end{document}
